%% LaTeX template for BSc Computing for Games final year project dissertations
%% by Edward Powley
%% Games Academy, Falmouth University, UK

%% Based on:
%% bare_jrnl.tex
%% V1.4b
%% 2015/08/26
%% by Michael Shell
%% see http://www.michaelshell.org/
%% for current contact information.
%%
%% This is a skeleton file demonstrating the use of IEEEtran.cls
%% (requires IEEEtran.cls version 1.8b or later) with an IEEE
%% journal paper.
%%
%% Support sites:
%% http://www.michaelshell.org/tex/ieeetran/
%% http://www.ctan.org/pkg/ieeetran
%% and
%% http://www.ieee.org/

%%*************************************************************************
%% Legal Notice:
%% This code is offered as-is without any warranty either expressed or
%% implied; without even the implied warranty of MERCHANTABILITY or
%% FITNESS FOR A PARTICULAR PURPOSE! 
%% User assumes all risk.
%% In no event shall the IEEE or any contributor to this code be liable for
%% any damages or losses, including, but not limited to, incidental,
%% consequential, or any other damages, resulting from the use or misuse
%% of any information contained here.
%%
%% All comments are the opinions of their respective authors and are not
%% necessarily endorsed by the IEEE.
%%
%% This work is distributed under the LaTeX Project Public License (LPPL)
%% ( http://www.latex-project.org/ ) version 1.3, and may be freely used,
%% distributed and modified. A copy of the LPPL, version 1.3, is included
%% in the base LaTeX documentation of all distributions of LaTeX released
%% 2003/12/01 or later.
%% Retain all contribution notices and credits.
%% ** Modified files should be clearly indicated as such, including  **
%% ** renaming them and changing author support contact information. **
%%*************************************************************************


\documentclass[journal]{IEEEtran}

\usepackage{graphicx}
% Insert additional usepackage commands here

\begin{document}
%
% paper title
% Titles are generally capitalized except for words such as a, an, and, as,
% at, but, by, for, in, nor, of, on, or, the, to and up, which are usually
% not capitalized unless they are the first or last word of the title.
% Linebreaks \\ can be used within to get better formatting as desired.
% Do not put math or special symbols in the title.
\title{Comparing game tree search techniques for general videogame AI (GVGAI) Literature Review}
%
%
% author name
\author{Alastair Rayner - 1507516}

% The paper headers -- please do not change these, but uncomment one of them as appropriate
% Uncomment this one for COMP320
\markboth{COMP320: Research Review and Proposal}{COMP320: Research Review and Proposal}
% Uncomment this one for COMP360
% \markboth{COMP360: Dissertation}{COMP360: Dissertation}

% make the title area
\maketitle

% As a general rule, do not put math, special symbols or citations
% in the abstract or keywords.
\begin{abstract}
The abstract goes here.
\end{abstract}

\section{Introduction}
\IEEEPARstart{T}{his} Literature review will cover what questions I will be asking for my dissertation topic as well all the literature I have found that is related to my research questions.




\section{Research Questions}

%TODO: Simplify research questions
\begin{itemize}
    \item How does game tree search techniques compare for GVGAI?
    \item Where does GVGAI succeed best in set games?
    \item Where does each tree search technique do well in each game?
    \item What are the most challenging areas for GVGAI in the GVGAI competition?
\end{itemize}

\section{Literature Review}
\subsection{The General Video Game AI Competition}

%General Video Game AI
In most modern video games the AI is tailored specifically for that game and can't easily be modified for use in a different game type. However this is what GVGAI aims to create, an AI that can play any game.

There have been quite a few AI competitions before in video games, such as Unreal Tournament \cite{citationNeeded}, Super Mario Bros \cite{citationNeeded}, Starcraft \cite{citationNeeded} \cite{perez20162014}. 


%GVGAI Competiton
The GVG-AI Competition is a competition framework that proposes the challenge of creating controllers for general video game playing. The controllers must be able to play a wide veraiety of video games, many of them will be completetly unknown to the controller. This means the controller must have some general AI to discover the merchanics and goal of the game, so it can increase it's score and win the game. \cite{GVGAI, perez20162014}

The framework contains a library of 2D Java based video games some of which are based of classic arcade games, there are currently as of writing this, 62 games that AI controllers can be tested on.

%Limits of the framework
The controllers are allowed upto 40ms to compute the agents action(s) \cite{perez2016GVGAICompetition, GVGAI}.

\subsection{Goal Orientation}
This paper 
\cite{ross2014general}


\subsection{Analyzing the Robustness of General Video Game Playing Agents}
This paper 
\cite{perez2016analyzing}

\subsection{Efficient Implementation of Breadth First Search for General Video Game Playing}
This paper proposes an efficient implementation of Breath First search, however it only works well for deterministic game sets.
The paper proposes a method of BFS where a node that has already been visited in other nodes will not be expanded, this is stored in a hash function.
The algorithm uses hash codes to improve the efficiency and performance.

\cite{EfficientBFS}

\subsection{HyperHeuristic}
Hyper Heuristic methods are 
\cite{hyperHeurisicMendes}

\section{Conclusion}
The conclusion goes here.

% references section

\bibliographystyle{IEEEtran}
\bibliography{references}

% Appendices

\appendices
\section{First appendix}
Appendices are optional. Delete or comment out this part if you do not need them.

% that's all folks
\end{document}
